\documentclass[14pt]{article}

\usepackage[a4paper,
            bindingoffset=0in,
            left=1in,
            right=1in,
            top=1in,
            bottom=1in,
            footskip=.25in]{geometry}
\usepackage[utf8]{inputenc}
\usepackage{fontspec}
\setmainfont{times-new-roman}[
  ExternalLocation = fonts/,
  Extension = .ttf,
  UprightFont = *-regular,
  BoldFont = *-bold,
  ItalicFont = *-italic,
  BoldItalicFont = *-bold-italic,
]
\usepackage[]{amsmath}
\usepackage[]{tabularx}
\usepackage[ukrainian,english]{babel}
\usepackage[none]{hyphenat}
\usepackage{titlesec}

\renewcommand{\baselinestretch}{1.5}

\newcolumntype{Y}{>{\centering\arraybackslash}X}
\titleformat{\chapter}{\filcenter\huge\bfseries}{\thechapter}{1em}{}
\titleformat{\section}{}{}{1em}{}

\begin{document}

\fontsize{14}{17}\selectfont

\begin{center}
НАЦІОНАЛЬНИЙ ТЕХНІЧНИЙ УНІВЕРСИТЕТ УКРАЇНИ\\
«КИЇВСЬКИЙ ПОЛІТЕХНІЧНИЙ ІНСТИТУТ ІМЕНІ ІГОРЯ СІКОРСЬКОГО»\\

\fontsize{12}{14}\selectfont
\underline{Кафедра інформатики та програмної інженерії}\\

\fontsize{8}{10}\selectfont
(повна назва кафедри, циклової комісії)\\

\fontsize{18}{22}\selectfont
\vspace{0.5cm}
\textbf{КУРСОВИЙ ПРОЄКТ}\\
\fontsize{14}{17}\selectfont
\textit{з  “Методології інженерії програмного забезпечення”}\\
на тему: “Програмно-визначена віддалена памʼять у розподілених системах”\\
\end{center}
\vspace{0.2cm}
{\raggedleft
\begin{tabular}{l@{}}
Студента 1 курсу групи ІП-22мп\\
спеціальності: \textit{121, Інженерія}\\
\multicolumn{1}{c}{\textit{програмного забезпечення}}\\
Волобуєва Нікіти Олександровича\\
Керівник:  Марченко О.
\end{tabular}\par}
\vspace{0.5cm}
{\raggedleft
\begin{tabular}{l@{}}
Національна оцінка \rule{5cm}{0.15mm}\\
Кількість балів: \rule{2cm}{0.15mm} Оцінка:  ECTS \rule{2cm}{0.15mm}
\end{tabular}\par}
\vspace{0.2cm}
{\raggedleft
\begin{tabular}{r}
Члени комісії\quad$\underset{\text{\fontsize{8}{10}\selectfont (підпис)}}{\text{\underline{\hspace{3cm}}}}$\quad$\underset{\text{\fontsize{8}{10}\selectfont (вчене звання, науковий ступінь, прізвище та ініціали)}}{\text{\underline{\hspace{7cm}}}}$\\
$\underset{\text{\fontsize{8}{10}\selectfont (підпис)}}{\text{\underline{\hspace{3cm}}}}$\quad$\underset{\text{\fontsize{8}{10}\selectfont (вчене звання, науковий ступінь, прізвище та ініціали)}}{\text{\underline{\hspace{7cm}}}}$
\end{tabular}\par}

\vspace*{\fill}
\begin{center}
Київ - 2023 рік
\end{center}

\thispagestyle{empty}

\pagebreak

\begin{center}
\fontsize{14}{17}\selectfont
Національний технічний університет України “КПІ імені Ігоря Сікорського”\\
\fontsize{12}{14}\selectfont
(назва вищого навчального закладу)\\
Кафедра \underline{інформатики та програмної інженерії}\\
Дисципліна \underline{«Оброблення надвеликих масивів даних»}\\
Спеціальність 121 \underline{"Інженерія програмного забезпечення"}\\
\end{center}
Курс \underline{\hspace{1em}1\hspace{1em}} Група \underline{\hspace{1em}ІП-22мп\hspace{1em}} \hfill Семестр \underline{\hspace{0.5em}2\hspace{0.5em}}\\

\fontsize{14}{17}\selectfont
\begin{center}
\textbf{ЗАВДАННЯ}\\
\textbf{на курсовий проект студента}\\

\begin{tabularx}{\textwidth}{X c X}
    & Волобуєва Нікіти Олександровича &\\
    \hline
    & \fontsize{9}{11}\selectfont (прізвище, ім’я, по батькові) &
\end{tabularx}
\end{center}

\fontsize{14}{17}\selectfont
\noindent
\begin{tabularx}{\textwidth}{l X}
    1. Тема проекту\\
    Програмно-визначена віддалена памʼять у розподілених системах\\
    \hline
\end{tabularx}
\\
\begin{tabularx}{\textwidth}{l X}
2. Строк здачі студентом закінченої роботи & 06.06.23\\
\cline{2-2}
\end{tabularx}
\\
\begin{tabularx}{\textwidth}{l X}
    3. Вихідні дані до роботи\\
    \\
    \hline
\end{tabularx}
\\
\begin{tabularx}{\textwidth}{l X}
    4. Зміст розрахунково-пояснювальної записки (перелік питань, які\\підлягають розробці)\\
    Архітектура програмного рішення програмно-визначеної віддаленої\\памʼяті\\
\end{tabularx}
\\
\begin{tabularx}{\textwidth}{l X}
    5.  Перелік графічного матеріалу ( з точним зазначенням обов’язкових\\ креслень )\\
    \\
    \hline
\end{tabularx}
\\
\begin{tabularx}{\textwidth}{l X}
    6. Дата видачі завдання & 20.03.23\\
\cline{2-2}
\end{tabularx}

\thispagestyle{empty}

\pagebreak

\begin{center}
АНОТАЦІЯ\\
\end{center}


\sloppy
Курсова робота виконана на 20 сторінках,  складається зі вступу, двох розділів, поділених на підрозділи, висновків, списку використаних джерел (4 посиланя) і містить 2 рисунки. Об’єкт дослідження: віддалена памʼять у розподілених системах. Мета роботи: аналіз проблеми, існуючих реалізацій та розробка архітектури програмного рішення програмно-визначеної віддаленої памʼяті.

\textit{Ключові слова:} FAR MEMORY, РОЗПОДІЛЕНІ СИСТЕМИ, КОМПʼЮТЕРНІ МЕРЕЖІ, СТРУКТУРИ ДАНИХ, LINUX.

\thispagestyle{empty}

\pagebreak

\begin{center}
\fontsize{18}{22}\selectfont
\textbf{КАЛЕНДАРНИЙ ПЛАН}
\end{center}

\thispagestyle{empty}

\fontsize{14}{17}\selectfont
\begin{flushleft}
\begin{tabularx}{1.15\textwidth}{| p{1.25cm} | X | p{2.5cm} | p{2.5cm} |}
\hline
\multicolumn{1}{|p{1.25cm}}{\centering № п/п} & \multicolumn{1}{|Y}{\centering Назва етапів курсового проекту }& \multicolumn{1}{|p{2.5cm}}{\centering  Термін виконання етапів проекту } & \multicolumn{1}{|p{2.5cm}|}{\centering Підписи керівника, студента}\\
\hline
\multicolumn{1}{|p{1.25cm}|}{\centering 1.} & Отримання теми курсового проекту & 20.03.2023 & \\
\hline
\multicolumn{1}{|p{1.25cm}|}{\centering 2.} & Визначення основних задач курсової роботи  & 25.03.2023 & \\
\hline
\multicolumn{1}{|p{1.25cm}|}{\centering 3.} & Аналіз предметної області, що розглядається & 01.04.2023 & \\
\hline
\multicolumn{1}{|p{1.25cm}|}{\centering 4.} & Аналіз досліджень за темою та існуючих реалізацій & 05.04.2023 & \\
\hline
\multicolumn{1}{|p{1.25cm}|}{\centering 5.} & Розробка архітектури програмного рішення & 15.04.2023 & \\
\hline
\multicolumn{1}{|p{1.25cm}|}{\centering 6.} & Опис архітектури програмного рішення & 01.05.2023 & \\
\hline
\multicolumn{1}{|p{1.25cm}|}{\centering 7.} & Підготовка пояснювальної записки & 15.05.2023 & \\
\hline
\multicolumn{1}{|p{1.25cm}|}{\centering 8.} & Здача курсової роботи  на перевірку & 01.06.2023 & \\
\hline
\multicolumn{1}{|p{1.25cm}|}{\centering 9.} & Захист курсової роботи & 06.06.2023 & \\
\hline
&&&\\
\hline
&&&\\
\hline
&&&\\
\hline
\end{tabularx}

\vspace{1cm}
\begin{tabularx}{1.15\textwidth}{p{4.5cm} p{2cm} Y Y}
\centering Студент &&& Волобуєв Н.О.\\
& \fontsize{8}{11}\selectfont \centering \underline{(підпис)} && \fontsize{8}{11}\selectfont \centering (прізвище, ім’я, по батькові)\\
\end{tabularx}

\vspace{1cm}
\begin{tabularx}{1.15\textwidth}{p{4.5cm} p{2cm} Y Y}
\centering Керівник &&& Марченко О.\\
& \fontsize{8}{11}\selectfont \centering \underline{(підпис)} && \fontsize{8}{11}\selectfont \centering (прізвище, ім’я, по батькові)\\
\end{tabularx}
\end{flushleft}

"\rule{1cm}{0.15mm}" \rule{4cm}{0.15mm} 2023 р.

\pagebreak

\fontsize{14}{17}\selectfont
\begin{center}
ЗМІСТ
\end{center}

\renewcommand\contentsname{}
\tableofcontents

\thispagestyle{empty}

\pagebreak

\chapter{вступ}

    У сучасному світі дуже поширеним є хмарне програмне забезпечення, яке з кожним днем замінює собою або інтегрується у вигляді нового функціоналу у існуюче програмне забезпечення в усіх галузях використання. Центральним компонентом такого програмного забезпечення є його серверна частина, що обслуговує запити багатьох користувачів. Цей компонент обробляє велику кількість запитів від різних користувачів зазвичай виконуючи найбільш ресурсоємну частину роботи у порівнянні з частиною розміщенною на пристрої кінцевого користувача. Оскільки ці ресурси зазвичай обмежені можливостями обладнання, що використовується (чи бюджетом на оренду такого обладнання), то будь-яка оптимізація використання ресурсів призводить до можливості обробляти більшу кількість запитів та тому ж самому обладнанні (що в результаті знижує витрати). 
    Оператори великих центрів обробки даних вже великий час застосовують різні методи для підвищення ефективності використання ресурсів серверного обладнання. Так, наприклад, для ефективного використання ресурсів процесору використовується підхід “надмірної підписки” (oversubscription) обчислювального часу. Схожий метод використовується і при організації інфраструктури сховищ даних в додачу до компресії та дедублікації даних.
Якщо перейти до ефективності використання оперативної памʼяті, то оператори найбільших у світі центрів обробки даних зазначають, що середнє використання памʼяті знаходиться на рівні близько 60\%. Для того, щоб покращити цей показник розробляються різні методи. Одним з цих методів є використання віддаленої памʼяті (Far Memory).
Cервери у центрі обробки данних (і програмне забезпечення, що на них розгорнуте) можна поділити на два типи: 
сервери, на яких більша частина памʼяті є вільною
сервери, які могли б цю памʼять використовувати, якщо мали би до неї доступ. 
    Суть методу віддаленої памʼяті полягає в тому, що сервери з вільною памʼяттю можуть надавати доступ до неї по мережі тому програмному забезпеченню, яке могло б її використовувати для зберігання тієї частини даних, що підходить для зберігання за умов та обмежень, що накладає віддалена памʼять.
    Завданням цього курсового проекту було поставлено аналіз проблеми, її існуючих рішень, формалізація методу та розробка архітектури програмного забезпечення для надання програмно-визначеної віддаленої памʼяті у розподілених системах.

\pagebreak

\chapter{РОЗДІЛ І. АНАЛІЗ ПРОБЛЕМИ}
\section{Ресурси обладнання у розподілених системах та проблема їх ефективного використання}

\end{document}