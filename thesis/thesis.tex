\documentclass[14pt]{article}

\usepackage[a4paper,
            bindingoffset=0.2in,
            left=1in,
            right=1in,
            top=1in,
            bottom=1in,
            footskip=.25in]{geometry}
\usepackage[utf8]{inputenc}
\usepackage{fontspec}
\setmainfont{times-new-roman}[
  ExternalLocation = fonts/,
  Extension = .ttf,
  UprightFont = *-regular,
  BoldFont = *-bold,
  ItalicFont = *-italic,
  BoldItalicFont = *-bold-italic,
]
\usepackage[]{amsmath}
\usepackage[]{tabularx}
\usepackage[ukrainian,english]{babel}
\usepackage[none]{hyphenat}

\renewcommand{\baselinestretch}{1.5}

\begin{document}

\fontsize{14}{17}\selectfont

\begin{center}
НАЦІОНАЛЬНИЙ ТЕХНІЧНИЙ УНІВЕРСИТЕТ УКРАЇНИ\\
«КИЇВСЬКИЙ ПОЛІТЕХНІЧНИЙ ІНСТИТУТ ІМЕНІ ІГОРЯ СІКОРСЬКОГО»\\

\fontsize{12}{14}\selectfont
\underline{Кафедра інформатики та програмної інженерії}\\

\fontsize{8}{10}\selectfont
(повна назва кафедри, циклової комісії)\\

\fontsize{18}{22}\selectfont
\vspace{0.5cm}
\textbf{КУРСОВИЙ ПРОЄКТ}\\
\fontsize{14}{17}\selectfont
\textit{з  “Методології інженерії програмного забезпечення”}\\
на тему: “Програмно-визначена віддалена памʼять у розподілених системах”\\
\end{center}
\vspace{0.2cm}
{\raggedleft
\begin{tabular}{l@{}}
Студента 1 курсу групи ІП-22мп\\
спеціальності: \textit{121, Інженерія}\\
\multicolumn{1}{c}{\textit{програмного забезпечення}}\\
Волобуєва Нікіти Олександровича\\
Керівник:  Марченко О.
\end{tabular}\par}
\vspace{0.5cm}
{\raggedleft
\begin{tabular}{l@{}}
Національна оцінка \rule{5cm}{0.15mm}\\
Кількість балів: \rule{2cm}{0.15mm} Оцінка:  ECTS \rule{2cm}{0.15mm}
\end{tabular}\par}
\vspace{0.2cm}
{\raggedleft
\begin{tabular}{r}
Члени комісії\quad$\underset{\text{\fontsize{8}{10}\selectfont (підпис)}}{\text{\underline{\hspace{3cm}}}}$\quad$\underset{\text{\fontsize{8}{10}\selectfont (вчене звання, науковий ступінь, прізвище та ініціали)}}{\text{\underline{\hspace{7cm}}}}$\\
$\underset{\text{\fontsize{8}{10}\selectfont (підпис)}}{\text{\underline{\hspace{3cm}}}}$\quad$\underset{\text{\fontsize{8}{10}\selectfont (вчене звання, науковий ступінь, прізвище та ініціали)}}{\text{\underline{\hspace{7cm}}}}$
\end{tabular}\par}

\vspace*{\fill}
\begin{center}
Київ - 2023 рік
\end{center}

\thispagestyle{empty}

\pagebreak

\begin{center}
\fontsize{14}{17}\selectfont
Національний технічний університет України “КПІ імені Ігоря Сікорського”\\
\fontsize{12}{14}\selectfont
(назва вищого навчального закладу)\\
Кафедра \underline{інформатики та програмної інженерії}\\
Дисципліна \underline{«Оброблення надвеликих масивів даних»}\\
Спеціальність 121 \underline{"Інженерія програмного забезпечення"}\\
\end{center}
Курс \underline{\hspace{1em}1\hspace{1em}} Група \underline{\hspace{1em}ІП-22мп\hspace{1em}} \hfill Семестр \underline{\hspace{0.5em}2\hspace{0.5em}}\\

\fontsize{14}{17}\selectfont
\begin{center}
\textbf{ЗАВДАННЯ}\\
\textbf{на курсовий проект студента}\\

\begin{tabularx}{\textwidth}{X c X}
    & Волобуєва Нікіти Олександровича &\\
    \hline
    & \fontsize{9}{11}\selectfont (прізвище, ім’я, по батькові) &
\end{tabularx}
\end{center}

\fontsize{14}{17}\selectfont
\noindent
\begin{tabularx}{\textwidth}{l X}
    1. Тема проекту\\
    Програмно-визначена віддалена памʼять у розподілених системах\\
    \hline
\end{tabularx}
\\
\begin{tabularx}{\textwidth}{l X}
2. Строк здачі студентом закінченої роботи & 06.06.23\\
\cline{2-2}
\end{tabularx}
\\
\begin{tabularx}{\textwidth}{l X}
    3. Вихідні дані до роботи\\
    \\
    \hline
\end{tabularx}
\\
\begin{tabularx}{\textwidth}{l X}
    4. Зміст розрахунково-пояснювальної записки (перелік питань, які\\підлягають розробці)\\
    Архітектура програмного рішення програмно-визначеної віддаленої\\памʼяті\\
\end{tabularx}
\\
\begin{tabularx}{\textwidth}{l X}
    5.  Перелік графічного матеріалу ( з точним зазначенням обов’язкових\\ креслень )\\
    \\
    \hline
\end{tabularx}
\\
\begin{tabularx}{\textwidth}{l X}
    6. Дата видачі завдання & 20.03.23\\
\cline{2-2}
\end{tabularx}

\thispagestyle{empty}

\pagebreak

\begin{center}
АНОТАЦІЯ\\
\end{center}


\sloppy
Курсова робота виконана на 20 сторінках,  складається зі вступу, двох розділів, поділених на підрозділи, висновків, списку використаних джерел (4 посиланя) і містить 2 рисунки. Об’єкт дослідження: віддалена памʼять у розподілених системах. Мета роботи: аналіз проблеми, існуючих реалізацій та розробка архітектури програмного рішення програмно-визначеної віддаленої памʼяті.

\textit{Ключові слова:} FAR MEMORY, РОЗПОДІЛЕНІ СИСТЕМИ, КОМПʼЮТЕРНІ МЕРЕЖІ, СТРУКТУРИ ДАНИХ, LINUX.

\thispagestyle{empty}

\end{document}